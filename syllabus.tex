% Options for packages loaded elsewhere
\PassOptionsToPackage{unicode}{hyperref}
\PassOptionsToPackage{hyphens}{url}
\PassOptionsToPackage{dvipsnames,svgnames,x11names}{xcolor}
%
\documentclass[
  11pt,
]{article}

\usepackage{amsmath,amssymb}
\usepackage{iftex}
\ifPDFTeX
  \usepackage[T1]{fontenc}
  \usepackage[utf8]{inputenc}
  \usepackage{textcomp} % provide euro and other symbols
\else % if luatex or xetex
  \usepackage{unicode-math}
  \defaultfontfeatures{Scale=MatchLowercase}
  \defaultfontfeatures[\rmfamily]{Ligatures=TeX,Scale=1}
\fi
\usepackage{lmodern}
\ifPDFTeX\else  
    % xetex/luatex font selection
\fi
% Use upquote if available, for straight quotes in verbatim environments
\IfFileExists{upquote.sty}{\usepackage{upquote}}{}
\IfFileExists{microtype.sty}{% use microtype if available
  \usepackage[]{microtype}
  \UseMicrotypeSet[protrusion]{basicmath} % disable protrusion for tt fonts
}{}
\makeatletter
\@ifundefined{KOMAClassName}{% if non-KOMA class
  \IfFileExists{parskip.sty}{%
    \usepackage{parskip}
  }{% else
    \setlength{\parindent}{0pt}
    \setlength{\parskip}{6pt plus 2pt minus 1pt}}
}{% if KOMA class
  \KOMAoptions{parskip=half}}
\makeatother
\usepackage{xcolor}
\usepackage[margin=1.5in]{geometry}
\setlength{\emergencystretch}{3em} % prevent overfull lines
\setcounter{secnumdepth}{-\maxdimen} % remove section numbering
% Make \paragraph and \subparagraph free-standing
\makeatletter
\ifx\paragraph\undefined\else
  \let\oldparagraph\paragraph
  \renewcommand{\paragraph}{
    \@ifstar
      \xxxParagraphStar
      \xxxParagraphNoStar
  }
  \newcommand{\xxxParagraphStar}[1]{\oldparagraph*{#1}\mbox{}}
  \newcommand{\xxxParagraphNoStar}[1]{\oldparagraph{#1}\mbox{}}
\fi
\ifx\subparagraph\undefined\else
  \let\oldsubparagraph\subparagraph
  \renewcommand{\subparagraph}{
    \@ifstar
      \xxxSubParagraphStar
      \xxxSubParagraphNoStar
  }
  \newcommand{\xxxSubParagraphStar}[1]{\oldsubparagraph*{#1}\mbox{}}
  \newcommand{\xxxSubParagraphNoStar}[1]{\oldsubparagraph{#1}\mbox{}}
\fi
\makeatother


\providecommand{\tightlist}{%
  \setlength{\itemsep}{0pt}\setlength{\parskip}{0pt}}\usepackage{longtable,booktabs,array}
\usepackage{calc} % for calculating minipage widths
% Correct order of tables after \paragraph or \subparagraph
\usepackage{etoolbox}
\makeatletter
\patchcmd\longtable{\par}{\if@noskipsec\mbox{}\fi\par}{}{}
\makeatother
% Allow footnotes in longtable head/foot
\IfFileExists{footnotehyper.sty}{\usepackage{footnotehyper}}{\usepackage{footnote}}
\makesavenoteenv{longtable}
\usepackage{graphicx}
\makeatletter
\newsavebox\pandoc@box
\newcommand*\pandocbounded[1]{% scales image to fit in text height/width
  \sbox\pandoc@box{#1}%
  \Gscale@div\@tempa{\textheight}{\dimexpr\ht\pandoc@box+\dp\pandoc@box\relax}%
  \Gscale@div\@tempb{\linewidth}{\wd\pandoc@box}%
  \ifdim\@tempb\p@<\@tempa\p@\let\@tempa\@tempb\fi% select the smaller of both
  \ifdim\@tempa\p@<\p@\scalebox{\@tempa}{\usebox\pandoc@box}%
  \else\usebox{\pandoc@box}%
  \fi%
}
% Set default figure placement to htbp
\def\fps@figure{htbp}
\makeatother
% definitions for citeproc citations
\NewDocumentCommand\citeproctext{}{}
\NewDocumentCommand\citeproc{mm}{%
  \begingroup\def\citeproctext{#2}\cite{#1}\endgroup}
\makeatletter
 % allow citations to break across lines
 \let\@cite@ofmt\@firstofone
 % avoid brackets around text for \cite:
 \def\@biblabel#1{}
 \def\@cite#1#2{{#1\if@tempswa , #2\fi}}
\makeatother
\newlength{\cslhangindent}
\setlength{\cslhangindent}{1.5em}
\newlength{\csllabelwidth}
\setlength{\csllabelwidth}{3em}
\newenvironment{CSLReferences}[2] % #1 hanging-indent, #2 entry-spacing
 {\begin{list}{}{%
  \setlength{\itemindent}{0pt}
  \setlength{\leftmargin}{0pt}
  \setlength{\parsep}{0pt}
  % turn on hanging indent if param 1 is 1
  \ifodd #1
   \setlength{\leftmargin}{\cslhangindent}
   \setlength{\itemindent}{-1\cslhangindent}
  \fi
  % set entry spacing
  \setlength{\itemsep}{#2\baselineskip}}}
 {\end{list}}
\usepackage{calc}
\newcommand{\CSLBlock}[1]{\hfill\break\parbox[t]{\linewidth}{\strut\ignorespaces#1\strut}}
\newcommand{\CSLLeftMargin}[1]{\parbox[t]{\csllabelwidth}{\strut#1\strut}}
\newcommand{\CSLRightInline}[1]{\parbox[t]{\linewidth - \csllabelwidth}{\strut#1\strut}}
\newcommand{\CSLIndent}[1]{\hspace{\cslhangindent}#1}

\renewcommand\toprule[2]\relax
\renewcommand\bottomrule[2]\relax
\makeatletter
\@ifpackageloaded{tcolorbox}{}{\usepackage[skins,breakable]{tcolorbox}}
\@ifpackageloaded{fontawesome5}{}{\usepackage{fontawesome5}}
\definecolor{quarto-callout-color}{HTML}{909090}
\definecolor{quarto-callout-note-color}{HTML}{0758E5}
\definecolor{quarto-callout-important-color}{HTML}{CC1914}
\definecolor{quarto-callout-warning-color}{HTML}{EB9113}
\definecolor{quarto-callout-tip-color}{HTML}{00A047}
\definecolor{quarto-callout-caution-color}{HTML}{FC5300}
\definecolor{quarto-callout-color-frame}{HTML}{acacac}
\definecolor{quarto-callout-note-color-frame}{HTML}{4582ec}
\definecolor{quarto-callout-important-color-frame}{HTML}{d9534f}
\definecolor{quarto-callout-warning-color-frame}{HTML}{f0ad4e}
\definecolor{quarto-callout-tip-color-frame}{HTML}{02b875}
\definecolor{quarto-callout-caution-color-frame}{HTML}{fd7e14}
\makeatother
\makeatletter
\@ifpackageloaded{caption}{}{\usepackage{caption}}
\AtBeginDocument{%
\ifdefined\contentsname
  \renewcommand*\contentsname{Table of contents}
\else
  \newcommand\contentsname{Table of contents}
\fi
\ifdefined\listfigurename
  \renewcommand*\listfigurename{List of Figures}
\else
  \newcommand\listfigurename{List of Figures}
\fi
\ifdefined\listtablename
  \renewcommand*\listtablename{List of Tables}
\else
  \newcommand\listtablename{List of Tables}
\fi
\ifdefined\figurename
  \renewcommand*\figurename{Figure}
\else
  \newcommand\figurename{Figure}
\fi
\ifdefined\tablename
  \renewcommand*\tablename{Table}
\else
  \newcommand\tablename{Table}
\fi
}
\@ifpackageloaded{float}{}{\usepackage{float}}
\floatstyle{ruled}
\@ifundefined{c@chapter}{\newfloat{codelisting}{h}{lop}}{\newfloat{codelisting}{h}{lop}[chapter]}
\floatname{codelisting}{Listing}
\newcommand*\listoflistings{\listof{codelisting}{List of Listings}}
\makeatother
\makeatletter
\makeatother
\makeatletter
\@ifpackageloaded{caption}{}{\usepackage{caption}}
\@ifpackageloaded{subcaption}{}{\usepackage{subcaption}}
\makeatother
\makeatletter
\@ifpackageloaded{sidenotes}{}{\usepackage{sidenotes}}
\@ifpackageloaded{marginnote}{}{\usepackage{marginnote}}
\makeatother
\makeatletter
\@ifpackageloaded{fontawesome5}{}{\usepackage{fontawesome5}}
\makeatother

\usepackage{bookmark}

\IfFileExists{xurl.sty}{\usepackage{xurl}}{} % add URL line breaks if available
\urlstyle{same} % disable monospaced font for URLs
\hypersetup{
  pdftitle={ENGS 199.20 Decision Analysis for Wicked Climate Problems Syllabus},
  colorlinks=true,
  linkcolor={blue},
  filecolor={Maroon},
  citecolor={Blue},
  urlcolor={Blue},
  pdfcreator={LaTeX via pandoc}}


\title{ENGS 199.20 Decision Analysis for Wicked Climate Problems
Syllabus}
\usepackage{etoolbox}
\makeatletter
\providecommand{\subtitle}[1]{% add subtitle to \maketitle
  \apptocmd{\@title}{\par {\large #1 \par}}{}{}
}
\makeatother
\subtitle{Fall 2025}
\author{}
\date{}

\begin{document}
\maketitle


\subsection{Course Information}\label{course-information}

\subsubsection{Instructor}\label{instructor}

\begin{itemize}
\tightlist
\item
  \faIcon{user} \href{https://abpoll.github.io}{Adam Pollack}
\item
  \faIcon{envelope}
  \href{mailto:adam.b.pollack@dartmouth.edu}{\nolinkurl{adam.b.pollack@dartmouth.edu}}
\item
  \faIcon{building} IR 150
\item
  \faIcon{door-open} Mon. \& Wed., 2-3:30pm
\end{itemize}

\subsubsection{Meetings}\label{meetings}

\begin{itemize}
\tightlist
\item
  \faIcon{calendar} MWF
\item
  \faIcon{clock} 10:10-11:15am
\item
  \faIcon{calendar-times} Tues., 12:15pm - 1:05pm
\item
  \faIcon{university} TBD
\end{itemize}

\subsection{Course Description}\label{course-description}

The 21st century's biggest challenges resist traditional problem-solving
approaches. These wicked problems (Rittel \& Webber,
1973)\marginpar{\begin{footnotesize}
\begin{CSLReferences}{2}{0}
\bibitem[\citeproctext]{ref-rittel-webber-1973}
Rittel, H. W. J., \& Webber, M. M. (1973). Dilemmas in a general theory
of planning. \emph{Policy Sci.}, \emph{4}(2), 155--169.
\url{https://doi.org/10.1007/BF01405730}
\end{CSLReferences}
\vspace{2mm}\par\end{footnotesize}},
such as addressing climate risks, can't be solved with standard
engineering methods because they involve multiple stakeholders with
competing goals, uncertainties that evolve over time, and complex system
behaviors we're still trying to understand. Decision-makers need new
tools and frameworks that can handle this complexity while still
pointing the way toward practical action. In this course, students will
learn about state-of-the-art technical methods that offer promise for
meeting decision-making needs to respond to climate risks. Topics range
from complex systems analysis to reinforcement learning, and
multi-objective robust decision-making.

\subsection{Course Goals}\label{course-goals}

This course prepares engineers to tackle wicked climate problems.
Students master in-demand quantitative techniques in multi-objective
robust decision-making, which build on a number of courses in the Thayer
catalog. Students also develop essential professional skills in software
implementation, data visualization, and science communication. Through
carefully designed projects addressing the real-world challenges that
students care about, participants will build a solid portfolio that
demonstrates mastery of both technical and communication skills vital to
modern engineering practice. The course creates natural pathways to
honors theses, graduate research, and professional opportunities.

\subsection{Learning Objectives}\label{learning-objectives}

Upon completing this course, students will be able to:

\begin{enumerate}
\def\labelenumi{\arabic{enumi}.}
\tightlist
\item
  Frame complex societal challenges in ways amenable to structured
  decision analysis while preserving essential complexities.
\item
  Apply multi-objective robust decision-making frameworks to real-world
  problems using open source software tools.
\item
  Evaluate trade-offs between competing objectives using appropriate
  quantitative techniques.
\item
  Identify actionable insights for addressing problems characterized by
  deep uncertainties through sensitivity analysis and exploratory
  modeling.
\item
  Communicate complex technical analyses clearly and effectively through
  data visualization and presentation of trade-offs.
\item
  Develop professional-quality deliverables including software
  repositories, technical reports, and oral presentations.
\end{enumerate}

\subsection{Prerequisites}\label{prerequisites}

ENGS 93 or comparable course in probability and statistics. Students
should be proficient in a programming language such as Julia, Python, R,
or MATLAB. ENGS 172 recommended. \emph{The prerequisites can be replaced
by permission from the instructor.} If you are unsure if your background
is sufficient, please contact the instructor.

\begin{tcolorbox}[enhanced jigsaw, opacitybacktitle=0.6, colbacktitle=quarto-callout-tip-color!10!white, title=\textcolor{quarto-callout-tip-color}{\faLightbulb}\hspace{0.5em}{What If My Skills Are Rusty?}, coltitle=black, opacityback=0, rightrule=.15mm, titlerule=0mm, colback=white, colframe=quarto-callout-tip-color-frame, toprule=.15mm, breakable, arc=.35mm, bottomtitle=1mm, leftrule=.75mm, left=2mm, bottomrule=.15mm, toptitle=1mm]

If your programming or statistics skills are a little rusty, don't
worry! We will review concepts and build skills as needed. While we will
use the Python programming language in class, if you are familiar with
Julia or MATLAB, the fundamentals are similar.

\end{tcolorbox}

\subsection{Teaching Approach}\label{teaching-approach}

\subsubsection{Course Structure}\label{course-structure}

The course is divided into several modules and each week is generally
structured as follows:

\begin{itemize}
\tightlist
\item
  Mondays: Lecture
\item
  Tuesdays: Occasional X-hour for programming tutorials
\item
  Wednesdays: Student-led presentations or serious game followed by a
  group discussion
\item
  Fridays: Computational labs
\end{itemize}

\subsubsection{Expectations of Students}\label{expectations-of-students}

This course will require your sustained attention. Students are expected
to:

\begin{itemize}
\tightlist
\item
  come prepared to class (e.g., by carefully reading and synthesizing
  the reading assignments before class and being ready to present their
  synthesis in class);
\item
  actively contribute to the group discussions and activities;
\item
  submit the assignments on time; and
\item
  attend office hours as needed.
\end{itemize}

Students should expect to spend roughly three times the in-course hours
outside the classroom for readings and assignments.

\paragraph{Communication}\label{communication}

\begin{tcolorbox}[enhanced jigsaw, opacitybacktitle=0.6, colbacktitle=quarto-callout-important-color!10!white, title=\textcolor{quarto-callout-important-color}{\faExclamation}\hspace{0.5em}{Please, Be Excellent To Teach Other}, coltitle=black, opacityback=0, rightrule=.15mm, titlerule=0mm, colback=white, colframe=quarto-callout-important-color-frame, toprule=.15mm, breakable, arc=.35mm, bottomtitle=1mm, leftrule=.75mm, left=2mm, bottomrule=.15mm, toptitle=1mm]

We all make mistakes in our communications with one another, both when
speaking and listening. Be mindful of how spoken or written language
might be misunderstood, and be aware that, for a variety of reasons, how
others perceive your words and actions may not be exactly how you
intended them. At the same time, it is also essential that we be
respectful and interpret each other's comments and actions in good
faith.

\end{tcolorbox}

\begin{tcolorbox}[enhanced jigsaw, opacitybacktitle=0.6, colbacktitle=quarto-callout-important-color!10!white, title=\textcolor{quarto-callout-important-color}{\faExclamation}\hspace{0.5em}{Troubleshooting Tips}, coltitle=black, opacityback=0, rightrule=.15mm, titlerule=0mm, colback=white, colframe=quarto-callout-important-color-frame, toprule=.15mm, breakable, arc=.35mm, bottomtitle=1mm, leftrule=.75mm, left=2mm, bottomrule=.15mm, toptitle=1mm]

\begin{itemize}
\tightlist
\item
  \textbf{Do not take screenshots of code}. I will not respond.
  Screenshots can be difficult to read and limit accessibility. Put your
  code on GitHub, share the link, and point to specific line numbers if
  relevant, or provide a \emph{simple}, self-contained example of the
  problem you're running into.
\item
  If you wait until the day an assignment is due (or even late the
  previous night) to ask a question on Canvas, there is a strong chance
  that I will not see your post prior to the deadline.
\item
  If you see unanswered questions and you have some insight, please
  answer! This class will work best when we all work together as a
  community.
\end{itemize}

\end{tcolorbox}

\subsection{Assessments}\label{assessments}

\subsubsection{Active Participation:
10\%}\label{active-participation-10}

Collaboration and collegiality are important skills for
interdisciplinary decision analysts. We will practice these skills in
the following ways:

\begin{itemize}
\tightlist
\item
  Asking questions during course meetings and/or office hours;
\item
  Participating in serious games and group discussions;
\item
  Providing constructive feedback on student-led presentations;
\item
  Collaborating on computational labs and managing a shared GitHub
  repository.
\end{itemize}

Graduate students will additionally practice these skills by leading a
journal club session on one of the assigned readings.

Note that passively attending class will not yield full participation
points. Participation points are not free. I will record lectures,
student-led presentations and group discussions, and lab tutorials and
post them to Canvas. It is possible to attend these sessions remotely on
Zoom, but this can affect your ability to actively participate. It is
not possible to attend some of the Wednesday active learning sessions
remotely and these will not be recorded.

\subsubsection{Computational Labs: 30\%}\label{computational-labs-30}

We will use class time for hands-on programming exercises to give you
guided practice applying the concepts and methods from class. These labs
will be done in groups; if you cannot bring a laptop to class, you will
be able to (and are encouraged to) work with other students, though you
must turn in your own lab report for grading. All labs will cover either
a set of programming tools or a case study, such as reproducing analyses
in peer-reviewed studies (e.g.,
\url{https://github.com/abpoll/j40_gc/tree/v1.0.0-pub} and
\url{https://github.com/jdossgollin/2022-elevation-robustness}).

Lab reports are due before the following lab.

\subsubsection{Course Project: 60\%}\label{course-project-60}

The goal of the course project is to develop and report on an actionable
plan for implementing a quantitative decision analysis for the student's
topic of choice. This project can be highly synergistic with thesis
projects and students are encouraged to choose a decision problem
related to their thesis if they are working on one.

Students will work on this project throughout the entire term, with
ongoing evaluations and three major evaluation milestones as follows:

\begin{itemize}
\tightlist
\item
  Project Progress Reports: At the end of each course module, students
  will have one week to submit a progress report that details a
  technically appropriate plan to integrate methods covered in the
  module in a quantitative decision analysis. A list of guiding
  questions will be available at the beginning of each module, so
  students will have more than a week to work on these reports. These
  reports test your understanding and synthesis of material in each
  module, and your ability to extend the underlying concepts and tools
  in an application area of your choice.
\item
  Project Presentation: Students will present their projects to the
  class at the end of the term. Presentations should be no more than 15
  minutes long and should be aimed at a general science audience. Each
  presentation should demonstrate comprehension of technical aspects,
  clearly communicate limitations and opportunities for future work, and
  include signposts for peer feedback. Part of your grade on this
  component consists of the quality of your feedback to your peers on
  their presentations (more on this in the relevant assignment).
\item
  Written Project Report: Students will synthesize their term-long
  progress reports and feedback from their project presentation into a
  detailed proposal for how to implement the planned decision analysis.
\end{itemize}

\textbf{These components are each worth 20\% of your overall grade. The
course project is broken into complimentary components to help keep you
on track over the term and to lower the stakes of the overall project.}

Assignments must be submitted via Canvas by the deadline to receive a
full grade. Late submissions will receive a downgrading by 25\% of the
full credit for each day they are late. Submissions after the end of the
examination period do not receive any points. Each partial day will be
rounded up. You may want to submit a few hours before the deadline to
avoid last minute technical problems. Extension requests must be made
via email 24 hours before the submission deadline with a submission of
what has been achieved, thus far.

\subsection{Class Resources}\label{class-resources}

This course covers material in a relatively new and fast evolving field.
As such, there is no single resource that comprehensively synthesizes
the state-of-the-art in decision analysis for wicked climate problems.
Students may find it helpful to consult the open access textbook
published several years ago with contributions from prominent scientists
in the field titled Decision Making under Deep Uncertainty\,: From
Theory to Practice. (Marchau et al.,
2019)\marginpar{\begin{footnotesize}
\begin{CSLReferences}{2}{0}
\bibitem[\citeproctext]{ref-marchau-etal-2019}
Marchau, V. A. W., Walker, W. E., Bloemen, P. J. T., \& Popper, S. W.
(Eds.). (2019). \emph{Decision making under deep uncertainty: From
theory to practice} (2019th ed.). Cham, Switzerland: Springer Nature.
\url{https://doi.org/10.1007/978-3-030-05252-2}
\end{CSLReferences}
\vspace{2mm}\par\end{footnotesize}}

Other course materials consist, for example, of meeting notes,
peer-reviewed publications, and reports. Students will be able to access
all materials freely through Canvas or through the Dartmouth library
system.

\subsection{Class Policies}\label{class-policies}

\subsubsection{Canvas}\label{canvas}

Course communication, assignments, submissions of written materials will
be handled through the course Canvas site. Recordings will be made for
most sessions (i.e., not several active learning sessions on Wednesdays)
and posted on Canvas.

\subsubsection{Academic Honor Principle}\label{academic-honor-principle}

Students are expected to follow
\href{https://policies.dartmouth.edu/policy/academic-honor-principle}{Dartmouth's
Academic Honor Principle} and
\href{https://engineering.dartmouth.edu/about/policies/student-policies/academic-honor-policy}{Thayer's
Academic Honor Policy}.

\subsubsection{Religious Observances}\label{religious-observances}

Dartmouth has a deep commitment to support students' religious
observances and diverse faith practices. Some students may wish to take
part in religious observances that occur during this academic term. If
you have a religious observance that conflicts with your participation
in the course, please meet with me as soon as possible --- before the
end of the second week of the term at the latest --- to discuss
appropriate course adjustments.

\subsubsection{Student Accessibility and
Accommodations}\label{student-accessibility-and-accommodations}

Students requesting disability-related accommodations and services for
this course are required to register with Student Accessibility Services
(SAS;
\href{https://students.dartmouth.edu/student-accessibility/students/where-start/apply-services}{Apply
for Services webpage};
\href{mailto:student.accessibility.services@dartmouth.edu}{\nolinkurl{student.accessibility.services@dartmouth.edu}};
1-603-646-9900) and to request that an accommodation email be sent to me
in advance of the need for an accommodation. Then, students should
schedule a follow-up meeting with me to determine relevant details such
as what role SAS or its
\href{https://students.dartmouth.edu/student-accessibility/services/testing-center}{Testing
Center} may play in accommodation implementation. This process works
best for everyone when completed as early in the quarter as possible. If
students have questions about whether they are eligible for
accommodations or have concerns about the implementation of their
accommodations, they should contact the SAS office. All inquiries and
discussions will remain confidential.

\subsubsection{Class Climate and
Inclusivity}\label{class-climate-and-inclusivity}

We will discuss topics that for some are associated with strong emotions
and opinions. We will read and discuss sources from a wide range of
authors. These sources are selected to sample a wide range of diverse
perspectives. However, it seems likely that important perspectives are
missing.

This class strives to provide a respectful, inclusive, and civil space
for all. Please inform all members of this class if your names and
pronouns you prefer differ from your official record. Please reach out
to the instructor if your performance and ability to learn is impacted
by factors outside of the classroom (including financial challenges).
The instructor is here to help as much as possible. Feedback can also be
anonymous (including a note under the office door of the instructor). We
all are continuously learning about how to ensure an inclusive
environment and how to discuss contentious and emotional subjects.
Please provide feedback if something makes you feel uncomfortable.

\subsubsection{Use of GenAI}\label{use-of-genai}

Work submitted for a grade in this course must reflect your own
understanding. The use and consultation of AI/ML tools, such as ChatGPT
or similar, for any purpose whatsoever, \textbf{must be pre-approved and
clearly referenced.} Please see
\href{https://policies.dartmouth.edu/policy/guidelines-using-generative-artificial-intelligence-genai-coursework}{Dartmouth's
Guidelines on using GenAI}.

My general view is that Large language models (LLMs) are powerful tools
that you will encounter and use when you leave this classroom, so it's
important to learn how to use them responsibly and effectively. However,
these are difficult skills to teach and are not the focus of this
course. I will likely \textbf{not} accept requests to use GenAI to
generate text for writing assignments. However, I may accept requests to
use LLMs to support coding tasks. This can help accelerate your
workflow, especially when you are less familiar with a new language.
However, LLMs can end up being a less productive use of your time than
talking to colleagues, checking out documentation and APIs, looking at
StackOverflow, and experimenting with trial and failure. You are
responsible for understanding and debugging your code, and for ensuring
that it does what you intend it to do.

If approved, you must:

\begin{itemize}
\tightlist
\item
  reference the URL of the service you are using, including the specific
  date you accessed it;
\item
  provide the exact query or queries used to interact with the tool; and
\item
  report the exact response received.
\end{itemize}

\subsubsection{Student's Consent}\label{students-consent}

By enrolling in this course:

\begin{enumerate}
\def\labelenumi{\arabic{enumi}.}
\item
  I affirm my understanding that the instructor may record meetings of
  this course and any associated meetings open to multiple students and
  the instructor, including but not limited to scheduled and ad hoc
  office hours and other consultations, within any digital platform,
  including those used to offer remote instruction for this course.
\item
  I further affirm that the instructor owns the copyright to their
  instructional materials, of which these recordings constitute a part,
  and my distribution of any of these recordings in whole or in part to
  any person or entity other than other members of the class without
  prior written consent of the instructor may be subject to discipline
  by Dartmouth up to and including separation from Dartmouth.
\item
  Requirement of consent to one-on-one recordings By enrolling in this
  course, I hereby affirm that I will not make a recording in any medium
  of any one-on-one meeting with the instructor or another member of the
  class or group of members of the class without obtaining the prior
  written consent of all those participating, and I understand that if I
  violate this prohibition, I will be subject to discipline by Dartmouth
  up to and including separation from Dartmouth, as well as any other
  civil or criminal penalties under applicable law. I understand that an
  exception to this consent applies to accommodations approved by SAS
  for a student's disability, and that one or more students in a class
  may record class lectures, discussions, lab sessions, and review
  sessions and take pictures of essential information, and/or be
  provided class notes for personal study use only.
\end{enumerate}

If you have questions, please contact the Office of the Dean of the
Faculty of Arts and Sciences.

\subsubsection{Mental Health and
Wellbeing}\label{mental-health-and-wellbeing}

The academic environment is challenging, our terms are intensive, and
classes are not the only demanding part of your life. There are a number
of resources available to you on campus to support your wellness,
including: the \href{https://www.dartmouth.edu/~chd/}{Counseling
Center}, which allows you to book triage appointments online, the
\href{https://students.dartmouth.edu/wellness-center/wellness-mindfulness/transition-resources-and-information/virtual-student-wellness-center}{Student
Wellness Center} which offers wellness check-ins, and your undergraduate
dean. The student-led
\href{https://journeys.dartmouth.edu/mentalhealthunion/peer-support/}{Dartmouth
Student Mental Health Union} and their peer support program may be
helpful if you would like to speak to a trained fellow student support
listener. If you need immediate assistance, please contact the counselor
on-call at (603) 646-9442 at any time. Please make me aware of anything
that will hinder your success in this course.

\subsubsection{Title IX}\label{title-ix}

At Dartmouth, we value integrity, responsibility, and respect for the
rights and interests of others, all central to our Principles of
Community. We are dedicated to establishing and maintaining a safe and
inclusive campus where all community members have equal access to
Dartmouth's educational and employment opportunities. We strive to
promote an environment of sexual respect, safety, and well-being.
Through the Sexual and Gender-Based Misconduct Policy (SMP), Dartmouth
demonstrates that sex and gender-based discrimination, sex and
gender-based harassment, sexual assault, dating violence, domestic
violence, stalking, etc., are not tolerated in our community. For more
information regarding Title IX and to access helpful resources, visit
\href{https://sexual-respect.dartmouth.edu/}{Title IX's website}. As a
faculty member, I am required to share disclosures of sexual or
gender-based misconduct with the Title IX office. If you have any
questions or want to explore support and assistance, please contact the
Title IX office at 603-646-0922 or
\href{mailto:TitleIX@dartmouth.edu.}{\nolinkurl{TitleIX@dartmouth.edu.}}
Speaking to Title IX does not automatically initiate a college
resolution. Instead, much of their work is around providing supportive
measures to ensure you can continue to engage in Dartmouth's programs
and activities.





\end{document}
